%%%%%%%%%%%%%%%%%%%%%%%%%%%%%%%%%%%%%%%%%%%%%%%%%%%%%%%%%%%%%%%%%%%%%%%%%%%%%%%%
%2345678901234567890123456789012345678901234567890123456789012345678901234567890
%        1         2         3         4         5         6         7         8

\documentclass[letterpaper, 12 pt, conference]{ieeeconf}  % Comment this line out
                                                          % if you need a4paper
%\documentclass[a4paper, 10pt, conference]{ieeeconf}      % Use this line for a4
                                                          % paper

\IEEEoverridecommandlockouts                              % This command is only
                                                          % needed if you want to
                                                          % use the \thanks command
\overrideIEEEmargins
% See the \addtolength command later in the file to balance the column lengths
% on the last page of the document



% The following packages can be found on http:\\www.ctan.org
\usepackage{graphicx} % for pdf, bitmapped graphics files
\usepackage{bm}
\usepackage[english]{babel}
\usepackage{blindtext}
\usepackage{amsfonts}
\newcommand{\uvec}[1]{\boldsymbol{\hat{\textbf{#1}}}}
%\usepackage{epsfig} % for postscript graphics files
%\usepackage{mathptmx} % assumes new font selection scheme installed
%\usepackage{times} % assumes new font selection scheme installed
%\usepackage{amsmath} % assumes amsmath package installed
%\usepackage{amssymb}  % assumes amsmath package installed

\title{\LARGE \bf
Autonomous Navigation over Rough Terrain
}

\author{Xi Chen, John Folkesson and Patric Jensfelt % <-this % stops a space
%\author{Priyanshu Gandhi$^{1}$ and Hemant Kumar$^{2}$% <-this % stops a space
%\thanks{}% <-this % stops a space
%\thanks{$^{1}$Priyanshu Gandhi, 13116053, Department of Electronics and Communication Engineering}%
%\thanks{$^{2}$Hemant Kumar, 13116031, Department of Electronics and Communication Engineering}%
}

\begin{document}

\maketitle
\thispagestyle{empty}
\pagestyle{empty}

%%%%%%%%%%%%%%%%%%%%%%%%%%%%%%%%%%%%%%%%%%%%%%%%%%%%%%%%%%%%%%%%%%%%%%%%%%%%%%%%
\begin{abstract}
\blindtext

\end{abstract}


%%%%%%%%%%%%%%%%%%%%%%%%%%%%%%%%%%%%%%%%%%%%%%%%%%%%%%%%%%%%%%%%%%%%%%%%%%%%%%%%
\section{Introduction (later)}
The objective of the navigation system is to successfully move the robot from the initial pose A to the target pose B with combined driving and stepping motion. In the case that target B is located in unknown area or there are vast unknown area between A and B, a sub-target pose B' will be selected.
The navigation system will extract traversability features from the environment and integrate all measureable information into a knowledge map. Using the current knowledge map, the navigation system will generate one or multiple paths  based on criteria such as environmental uncertainty, travelling time, safety and efficiency.

\section{Problem Formulation (done)}
The workspace $\mathcal{W} = \mathbb{R}^{N}$, in witch $N=2$ or $N=3$ represents the environment in 2D or 3D space. The closed set $\mathcal{O}\subset \mathcal{W}$ represents the obstacle region, which is usually expressed as a collection of polygon (in 2D space) or polyhedra (in 3D space). 
The configuration of the robot $\textbf{q}$ is a $k$-dimensional vector which spicily the configuration of the robot using $k$ real values.  
The configuration space or the C-space $\mathcal{C}$ represents the set of all transformations that can be applied to a robot given its kinematics. A configuration $\textbf{q}$ is a point in $\mathcal{C}$.

Let $\mathcal{A}$ represents a rigid body robot and the closed subset $\mathcal{A}(\textbf{q})$ of the workspace $\mathcal{W}$ that is occupied by a configuration $\textbf{q}$ of $\mathcal{A}$ is donated as $\mathcal{A}(\textbf{q})\subset \mathcal{W}$.
The obstacle region $\mathcal{C}_{obs}$ in $\mathcal{C}$ is defined as:
\[
\mathcal{C}_{obs} = \left \{\textbf{q} \in \mathcal{C} |\mathcal{A}(\textbf{q})\cap \mathcal{O} \neq \phi \right \}
\]
The set of configuration that avoid collision $\mathcal{C}_{free}$ is the complement of $\mathcal{C}_{obs}$ in $\mathcal{C}$.
\[
\mathcal{C}_{free} = \left \{\textbf{q} \in \mathcal{C} |\mathcal{A}(\textbf{q})\cap \mathcal{O} = \phi \right \}
\]

The path planning problem can be defined as: given a robot description $\mathcal{A}$, a C-space where $\mathcal{C}_{free}$ and $\mathcal{C}_{obs}$ are defined, an initial configuration $\textbf{q}_{I}\in \mathcal{C}_{free}$ and a goal configuration $\textbf{q}_{G} \in \mathcal{C}_{free}$, compute a continues path $\tau:\left [ 0,1 \right ]\rightarrow \mathcal{C}_{free}$ that connects the initial configuration and the goal configuration with $\tau(0)=\textbf{q}_{I}$ and $\tau(1)=\textbf{q}_{G}$. 

The path planning with only kinematic constraints is often referred to as nonholonomic planning. 
For some problems, there are constrains on the robot dynamics such as the velocity, acceleration and applied forces. Planning with dynamic constraints is known as kinodynamic planning. 
In the kinodynamic planning, the state space $\mathcal{S}$ is introduced to deal with dynamics constraints. A state $s$ encodes a configuration and the velocity of each components, $s = (\textbf{q}, \dot{\textbf{q}})$.

In this paper, we set our focus on the non-holonomic planning by assuming the robot operates at a low speed that dynamics are not considered during path planning. 

\section{Related Work (editing)}
It is not easy to directly compute $\mathcal{C}_{obs}$ and $\mathcal{C}_{free}$ in motion planning problem, and the dimensionality of the C-space is often quite high. There are two general strategies to approach the $\mathcal{C}_{obs}$ and $\mathcal{C}_{free}$ and generate a path:
\begin{enumerate}
\item The connectivity graph (sampling-based approach): a connectivity graph (roadmap) in C-space is first constructed and a path is generated by applying graph search.

(cannot provide the guarantees of a complete algorithm)

(probabilistic roadmap PRM, rapidly exploring random tree RRT, shortest-path roadmap, vertical cell decomposition, grid-based approaches)
\item The Potential field: a function representing the influence of a potential field obtained by superposing an attractive potential towards the goal and a repulsive potential from the obstacle. The gradient of this function can guide the robot to the goal. 

(local minima)
\end{enumerate}

approaches developed based on classical methods:

......
\cite{greenwade93} % for citation testing

\section{Method (editing)}
\section{World modeling}
\subsection{Terrain type}

\textit{safe}, \textit{risky}, \textit{obstacle}.

\subsection{Graph map}
Apply clattering and filtering to the raw terrain map.
 
Decompose the map into different region based on the terrain type and approximate the shape of the region using polygon.  

Construct a graph map based on the connecting relationship between regions.


\section{hierarchical path planning}
Generate a global path based on the graph map.

Generate detailed local path to traverse through regions.

\subsection{Global path}

\subsection{Planning on safe terrain}

The robot can be treated as a point. Transition cost between points is low. 

\subsection{Planning on risky terrain}

The robot orientation need to be considered. Transition cost is higher than traveling on safe terrain. 

\section{Experiment}

\section{Conclusion}


\bibliographystyle{unsrt}
\bibliography{bib}

\end{document}
